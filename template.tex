% Options for packages loaded elsewhere
\PassOptionsToPackage{unicode}{hyperref}
\PassOptionsToPackage{hyphens}{url}
%
\documentclass[
]{article}
\usepackage{amsmath,amssymb}
\usepackage{iftex}
\ifPDFTeX
  \usepackage[T1]{fontenc}
  \usepackage[utf8]{inputenc}
  \usepackage{textcomp} % provide euro and other symbols
\else % if luatex or xetex
  \usepackage{unicode-math} % this also loads fontspec
  \defaultfontfeatures{Scale=MatchLowercase}
  \defaultfontfeatures[\rmfamily]{Ligatures=TeX,Scale=1}
\fi
\usepackage{lmodern}
\ifPDFTeX\else
  % xetex/luatex font selection
\fi
% Use upquote if available, for straight quotes in verbatim environments
\IfFileExists{upquote.sty}{\usepackage{upquote}}{}
\IfFileExists{microtype.sty}{% use microtype if available
  \usepackage[]{microtype}
  \UseMicrotypeSet[protrusion]{basicmath} % disable protrusion for tt fonts
}{}
\makeatletter
\@ifundefined{KOMAClassName}{% if non-KOMA class
  \IfFileExists{parskip.sty}{%
    \usepackage{parskip}
  }{% else
    \setlength{\parindent}{0pt}
    \setlength{\parskip}{6pt plus 2pt minus 1pt}}
}{% if KOMA class
  \KOMAoptions{parskip=half}}
\makeatother
\usepackage{xcolor}
\usepackage[margin=1in]{geometry}
\usepackage{color}
\usepackage{fancyvrb}
\newcommand{\VerbBar}{|}
\newcommand{\VERB}{\Verb[commandchars=\\\{\}]}
\DefineVerbatimEnvironment{Highlighting}{Verbatim}{commandchars=\\\{\}}
% Add ',fontsize=\small' for more characters per line
\usepackage{framed}
\definecolor{shadecolor}{RGB}{248,248,248}
\newenvironment{Shaded}{\begin{snugshade}}{\end{snugshade}}
\newcommand{\AlertTok}[1]{\textcolor[rgb]{0.94,0.16,0.16}{#1}}
\newcommand{\AnnotationTok}[1]{\textcolor[rgb]{0.56,0.35,0.01}{\textbf{\textit{#1}}}}
\newcommand{\AttributeTok}[1]{\textcolor[rgb]{0.13,0.29,0.53}{#1}}
\newcommand{\BaseNTok}[1]{\textcolor[rgb]{0.00,0.00,0.81}{#1}}
\newcommand{\BuiltInTok}[1]{#1}
\newcommand{\CharTok}[1]{\textcolor[rgb]{0.31,0.60,0.02}{#1}}
\newcommand{\CommentTok}[1]{\textcolor[rgb]{0.56,0.35,0.01}{\textit{#1}}}
\newcommand{\CommentVarTok}[1]{\textcolor[rgb]{0.56,0.35,0.01}{\textbf{\textit{#1}}}}
\newcommand{\ConstantTok}[1]{\textcolor[rgb]{0.56,0.35,0.01}{#1}}
\newcommand{\ControlFlowTok}[1]{\textcolor[rgb]{0.13,0.29,0.53}{\textbf{#1}}}
\newcommand{\DataTypeTok}[1]{\textcolor[rgb]{0.13,0.29,0.53}{#1}}
\newcommand{\DecValTok}[1]{\textcolor[rgb]{0.00,0.00,0.81}{#1}}
\newcommand{\DocumentationTok}[1]{\textcolor[rgb]{0.56,0.35,0.01}{\textbf{\textit{#1}}}}
\newcommand{\ErrorTok}[1]{\textcolor[rgb]{0.64,0.00,0.00}{\textbf{#1}}}
\newcommand{\ExtensionTok}[1]{#1}
\newcommand{\FloatTok}[1]{\textcolor[rgb]{0.00,0.00,0.81}{#1}}
\newcommand{\FunctionTok}[1]{\textcolor[rgb]{0.13,0.29,0.53}{\textbf{#1}}}
\newcommand{\ImportTok}[1]{#1}
\newcommand{\InformationTok}[1]{\textcolor[rgb]{0.56,0.35,0.01}{\textbf{\textit{#1}}}}
\newcommand{\KeywordTok}[1]{\textcolor[rgb]{0.13,0.29,0.53}{\textbf{#1}}}
\newcommand{\NormalTok}[1]{#1}
\newcommand{\OperatorTok}[1]{\textcolor[rgb]{0.81,0.36,0.00}{\textbf{#1}}}
\newcommand{\OtherTok}[1]{\textcolor[rgb]{0.56,0.35,0.01}{#1}}
\newcommand{\PreprocessorTok}[1]{\textcolor[rgb]{0.56,0.35,0.01}{\textit{#1}}}
\newcommand{\RegionMarkerTok}[1]{#1}
\newcommand{\SpecialCharTok}[1]{\textcolor[rgb]{0.81,0.36,0.00}{\textbf{#1}}}
\newcommand{\SpecialStringTok}[1]{\textcolor[rgb]{0.31,0.60,0.02}{#1}}
\newcommand{\StringTok}[1]{\textcolor[rgb]{0.31,0.60,0.02}{#1}}
\newcommand{\VariableTok}[1]{\textcolor[rgb]{0.00,0.00,0.00}{#1}}
\newcommand{\VerbatimStringTok}[1]{\textcolor[rgb]{0.31,0.60,0.02}{#1}}
\newcommand{\WarningTok}[1]{\textcolor[rgb]{0.56,0.35,0.01}{\textbf{\textit{#1}}}}
\usepackage{graphicx}
\makeatletter
\newsavebox\pandoc@box
\newcommand*\pandocbounded[1]{% scales image to fit in text height/width
  \sbox\pandoc@box{#1}%
  \Gscale@div\@tempa{\textheight}{\dimexpr\ht\pandoc@box+\dp\pandoc@box\relax}%
  \Gscale@div\@tempb{\linewidth}{\wd\pandoc@box}%
  \ifdim\@tempb\p@<\@tempa\p@\let\@tempa\@tempb\fi% select the smaller of both
  \ifdim\@tempa\p@<\p@\scalebox{\@tempa}{\usebox\pandoc@box}%
  \else\usebox{\pandoc@box}%
  \fi%
}
% Set default figure placement to htbp
\def\fps@figure{htbp}
\makeatother
\setlength{\emergencystretch}{3em} % prevent overfull lines
\providecommand{\tightlist}{%
  \setlength{\itemsep}{0pt}\setlength{\parskip}{0pt}}
\setcounter{secnumdepth}{-\maxdimen} % remove section numbering
\usepackage{bookmark}
\IfFileExists{xurl.sty}{\usepackage{xurl}}{} % add URL line breaks if available
\urlstyle{same}
\hypersetup{
  pdftitle={Visualizing the Uninsured: A Data Science Perspective on U.S. Health Coverage},
  hidelinks,
  pdfcreator={LaTeX via pandoc}}

\title{Visualizing the Uninsured: A Data Science Perspective on U.S.
Health Coverage}
\author{Eduardo Sáenz-Messía Laborda\\
Catherina Mikhail (2847118)\\
Charlotte Craenen (2865572)\\
Hugo van As\\
Indy Pleijter\\
Julian Zeguers\\
Wies Polderman\\
Tutorial group 5, group 4 -- J.F. Fitzgerald}
\date{2025-06-24}

\begin{document}
\maketitle

\section{Part 1 -- Identify a Social
Problem}\label{part-1-identify-a-social-problem}

\subsection{1.1 Describe the Social
Problem}\label{describe-the-social-problem}

The high number of uninsured Americans is a persistent issue with
substantial social and economic consequences. According to the U.S.
Census Bureau, over 27 million Americans lacked health insurance in
2022. This lack of coverage can lead to delayed medical care, poor
health outcomes, and financial instability.

\subsection{1.2 Provide Background on the
Problem}\label{provide-background-on-the-problem}

Since the implementation of the Affordable Care Act (ACA) in 2010, the
number of uninsured has decreased, but disparities remain. People with
low income, people of color, and those living in states that did not
expand Medicaid are disproportionately affected. Health insurance in the
U.S. is often tied to employment, further complicating access for those
in part-time, temporary, or informal jobs.

\section{Part 2 -- Describe and Acquire
Data}\label{part-2-describe-and-acquire-data}

\subsection{2.1 Describe the Dataset}\label{describe-the-dataset}

The dataset used is derived from the United States Census Bureau and
includes individual-level data on health insurance coverage,
demographics, and employment status.

\subsection{2.2 Import and Prepare the
Dataset}\label{import-and-prepare-the-dataset}

\begin{Shaded}
\begin{Highlighting}[]
\CommentTok{\# Example of data import}
\CommentTok{\# dataset \textless{}{-} read\_csv("data/uninsured\_2022.csv")}
\end{Highlighting}
\end{Shaded}

\section{Part 3 -- Visualize and Analyze the
Data}\label{part-3-visualize-and-analyze-the-data}

\subsection{3.1 Create Initial
Visualizations}\label{create-initial-visualizations}

\begin{Shaded}
\begin{Highlighting}[]
\CommentTok{\# ggplot(dataset, aes(x = age, fill = insurance\_status)) +}
\CommentTok{\#   geom\_histogram(binwidth = 5) +}
\CommentTok{\#   labs(title = "Insurance Status by Age", x = "Age", y = "Count")}
\end{Highlighting}
\end{Shaded}

\subsection{3.2 Identify Trends and
Patterns}\label{identify-trends-and-patterns}

The uninsured rate is higher among younger adults, Hispanic and Black
populations, and those with lower income or educational attainment.

\section{Part 4 -- Communicate
Findings}\label{part-4-communicate-findings}

\subsection{4.1 Summarize Key Insights}\label{summarize-key-insights}

Our analysis highlights systemic inequalities in access to health
insurance. Despite overall improvements since the ACA, millions remain
uninsured.

\subsection{4.2 Propose Solutions or Policy
Recommendations}\label{propose-solutions-or-policy-recommendations}

Potential solutions include:

\begin{itemize}
\tightlist
\item
  Expanding Medicaid in all states
\item
  Decoupling health insurance from employment
\item
  Increasing subsidies for marketplace plans
\end{itemize}

\section{Appendix}\label{appendix}

\subsection{A.1 References}\label{a.1-references}

\begin{itemize}
\tightlist
\item
  U.S. Census Bureau. (2022). Health Insurance Coverage in the United
  States.
\item
  Kaiser Family Foundation. (2023). Key Facts about the Uninsured
  Population.
\end{itemize}

\subsection{A.2 Session Info}\label{a.2-session-info}

\end{document}
