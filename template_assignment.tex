% Options for packages loaded elsewhere
\PassOptionsToPackage{unicode}{hyperref}
\PassOptionsToPackage{hyphens}{url}
%
\documentclass[
]{article}
\usepackage{amsmath,amssymb}
\usepackage{iftex}
\ifPDFTeX
  \usepackage[T1]{fontenc}
  \usepackage[utf8]{inputenc}
  \usepackage{textcomp} % provide euro and other symbols
\else % if luatex or xetex
  \usepackage{unicode-math} % this also loads fontspec
  \defaultfontfeatures{Scale=MatchLowercase}
  \defaultfontfeatures[\rmfamily]{Ligatures=TeX,Scale=1}
\fi
\usepackage{lmodern}
\ifPDFTeX\else
  % xetex/luatex font selection
\fi
% Use upquote if available, for straight quotes in verbatim environments
\IfFileExists{upquote.sty}{\usepackage{upquote}}{}
\IfFileExists{microtype.sty}{% use microtype if available
  \usepackage[]{microtype}
  \UseMicrotypeSet[protrusion]{basicmath} % disable protrusion for tt fonts
}{}
\makeatletter
\@ifundefined{KOMAClassName}{% if non-KOMA class
  \IfFileExists{parskip.sty}{%
    \usepackage{parskip}
  }{% else
    \setlength{\parindent}{0pt}
    \setlength{\parskip}{6pt plus 2pt minus 1pt}}
}{% if KOMA class
  \KOMAoptions{parskip=half}}
\makeatother
\usepackage{xcolor}
\usepackage[margin=1in]{geometry}
\usepackage{graphicx}
\makeatletter
\newsavebox\pandoc@box
\newcommand*\pandocbounded[1]{% scales image to fit in text height/width
  \sbox\pandoc@box{#1}%
  \Gscale@div\@tempa{\textheight}{\dimexpr\ht\pandoc@box+\dp\pandoc@box\relax}%
  \Gscale@div\@tempb{\linewidth}{\wd\pandoc@box}%
  \ifdim\@tempb\p@<\@tempa\p@\let\@tempa\@tempb\fi% select the smaller of both
  \ifdim\@tempa\p@<\p@\scalebox{\@tempa}{\usebox\pandoc@box}%
  \else\usebox{\pandoc@box}%
  \fi%
}
% Set default figure placement to htbp
\def\fps@figure{htbp}
\makeatother
\setlength{\emergencystretch}{3em} % prevent overfull lines
\providecommand{\tightlist}{%
  \setlength{\itemsep}{0pt}\setlength{\parskip}{0pt}}
\setcounter{secnumdepth}{-\maxdimen} % remove section numbering
\usepackage{bookmark}
\IfFileExists{xurl.sty}{\usepackage{xurl}}{} % add URL line breaks if available
\urlstyle{same}
\hypersetup{
  pdftitle={template\_assignment},
  hidelinks,
  pdfcreator={LaTeX via pandoc}}

\title{template\_assignment}
\author{}
\date{\vspace{-2.5em}}

\begin{document}
\maketitle

Template

Studentnames and studentnumbers here 2025-05-26 Set-up your environment
require(tidyverse) \#\# Loading required package: tidyverse \#\# --
Attaching core tidyverse packages ------------------------ tidyverse
2.0.0 -- \#\# v dplyr 1.1.4 v readr 2.1.5 \#\# v forcats 1.0.0 v stringr
1.5.1 \#\# v ggplot2 3.5.2 v tibble 3.2.1 \#\# v lubridate 1.9.4 v tidyr
1.3.1 \#\# v purrr 1.0.4 \#\# -- Conflicts
------------------------------------------ tidyverse\_conflicts() --
\#\# x dplyr::filter() masks stats::filter() \#\# x dplyr::lag() masks
stats::lag() \#\# i Use the conflicted package
(\url{http://conflicted.r-lib.org/}) to force all conflicts to become
errors Title Page Include your names Include the tutorial group number
Include your tutorial lecturer's name Part 1 - Identify a Social Problem
Use APA referencing throughout your document. 1.1 Describe the Social
Problem Include the following: • Why is this relevant? • . . . Part 2 -
Data Sourcing 2.1 Load in the data Preferably from a URL, but if not,
make sure to download the data and store it in a shared location that
you can load the data in from. Do not store the data in a folder you
include in the Github repository! 1 dataset \textless- midwest midwest
is an example dataset included in the tidyverse package 2.2 Provide a
short summary of the dataset(s) head(dataset) \#\# \# A tibble: 6 x 28
\#\# PID county state area poptotal popdensity popwhite popblack
popamerindian \#\# \#\# 1 561 ADAMS IL 0.052 66090 1271. 63917 1702 98
\#\# 2 562 ALEXAND\textasciitilde{} IL 0.014 10626 759 7054 3496 19 \#\#
3 563 BOND IL 0.022 14991 681. 14477 429 35 \#\# 4 564 BOONE IL 0.017
30806 1812. 29344 127 46 \#\# 5 565 BROWN IL 0.018 5836 324. 5264 547 14
\#\# 6 566 BUREAU IL 0.05 35688 714. 35157 50 65 \#\# \# i 19 more
variables: popasian , popother , percwhite , \#\# \# percblack ,
percamerindan , percasian , percother , \#\# \# popadults , perchsd ,
percollege , percprof , \#\# \# poppovertyknown , percpovertyknown ,
percbelowpoverty , \#\# \# percchildbelowpovert , percadultpoverty ,
\#\# \# percelderlypoverty , inmetro , category In this case we see 28
variables, but we miss some information on what units they are in. We
also don't know anything about the year/moment in which this data has
been captured. These are things that are usually included in the
metadata of the dataset. For your project, you need to provide us with
the information from your metadata that we need to understand your
dataset of choice. 2.3 Describe the type of variables included Think of
things like: • Do the variables contain health information or SES
information? • Have they been measured by interviewing individuals or is
the data coming from administrative sources? For the sake of this
example, I will continue with the assignment. . . Part 3 - Quantifying
3.1 Data cleaning Say we want to include only larger distances (above 2)
in our dataset, we can filter for this. mean(dataset\$percollege) \#\#
{[}1{]} 18.27274

Please use a separate `R block' of code for each type of cleaning. So,
e.g.~one for missing values, a new one for removing unnecessary
variables etc. 3.2 Generate necessary variables Variable 1 Variable 2 2
3.3 Visualize temporal variation 3.4 Visualize spatial variation Here
you provide a description of why the plot above is relevant to your
specific social problem. 3.5 Visualize sub-population variation What is
the poverty rate by state? \# Boxplot of poverty rate by state using the
`midwest' dataset ggplot(dataset, aes(x = state, y = percadultpoverty,
fill = state)) + geom\_boxplot() + labs( title = ``Distribution of
Poverty Rates by State (Midwest counties)'', x = ``State'', y =
``Poverty Rate of Adults (\%)'', fill = ``State'' ) + theme\_minimal() +
theme( legend.position = ``right'' ) 0 10 20 30 40 IL IN MI OH WI State
Poverty Rate of Adults (\%) State IL IN MI OH WI Distribution of Poverty
Rates by State (Midwest counties) Here you provide a description of why
the plot above is relevant to your specific social problem. 3.6 Event
analysis Analyze the relationship between two variables.

Here you provide a description of why the plot above is relevant to your
specific social problem. Part 4 - Discussion 4.1 Discuss your findings
Part 5 - Reproducibility 5.1 Github repository link Provide the link to
your PUBLIC repository here: . . . 5.2 Reference list Use APA
referencing throughout your document.

\end{document}
